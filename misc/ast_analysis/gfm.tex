% Options for packages loaded elsewhere
\PassOptionsToPackage{unicode}{hyperref}
\PassOptionsToPackage{hyphens}{url}
%
\documentclass[
]{article}
\usepackage{amsmath,amssymb}
\usepackage{iftex}
\ifPDFTeX
  \usepackage[T1]{fontenc}
  \usepackage[utf8]{inputenc}
  \usepackage{textcomp} % provide euro and other symbols
\else % if luatex or xetex
  \usepackage{unicode-math} % this also loads fontspec
  \defaultfontfeatures{Scale=MatchLowercase}
  \defaultfontfeatures[\rmfamily]{Ligatures=TeX,Scale=1}
\fi
\usepackage{lmodern}
\ifPDFTeX\else
  % xetex/luatex font selection
\fi
% Use upquote if available, for straight quotes in verbatim environments
\IfFileExists{upquote.sty}{\usepackage{upquote}}{}
\IfFileExists{microtype.sty}{% use microtype if available
  \usepackage[]{microtype}
  \UseMicrotypeSet[protrusion]{basicmath} % disable protrusion for tt fonts
}{}
\makeatletter
\@ifundefined{KOMAClassName}{% if non-KOMA class
  \IfFileExists{parskip.sty}{%
    \usepackage{parskip}
  }{% else
    \setlength{\parindent}{0pt}
    \setlength{\parskip}{6pt plus 2pt minus 1pt}}
}{% if KOMA class
  \KOMAoptions{parskip=half}}
\makeatother
\usepackage{xcolor}
\usepackage{color}
\usepackage{fancyvrb}
\newcommand{\VerbBar}{|}
\newcommand{\VERB}{\Verb[commandchars=\\\{\}]}
\DefineVerbatimEnvironment{Highlighting}{Verbatim}{commandchars=\\\{\}}
% Add ',fontsize=\small' for more characters per line
\newenvironment{Shaded}{}{}
\newcommand{\AlertTok}[1]{\textcolor[rgb]{1.00,0.00,0.00}{\textbf{#1}}}
\newcommand{\AnnotationTok}[1]{\textcolor[rgb]{0.38,0.63,0.69}{\textbf{\textit{#1}}}}
\newcommand{\AttributeTok}[1]{\textcolor[rgb]{0.49,0.56,0.16}{#1}}
\newcommand{\BaseNTok}[1]{\textcolor[rgb]{0.25,0.63,0.44}{#1}}
\newcommand{\BuiltInTok}[1]{\textcolor[rgb]{0.00,0.50,0.00}{#1}}
\newcommand{\CharTok}[1]{\textcolor[rgb]{0.25,0.44,0.63}{#1}}
\newcommand{\CommentTok}[1]{\textcolor[rgb]{0.38,0.63,0.69}{\textit{#1}}}
\newcommand{\CommentVarTok}[1]{\textcolor[rgb]{0.38,0.63,0.69}{\textbf{\textit{#1}}}}
\newcommand{\ConstantTok}[1]{\textcolor[rgb]{0.53,0.00,0.00}{#1}}
\newcommand{\ControlFlowTok}[1]{\textcolor[rgb]{0.00,0.44,0.13}{\textbf{#1}}}
\newcommand{\DataTypeTok}[1]{\textcolor[rgb]{0.56,0.13,0.00}{#1}}
\newcommand{\DecValTok}[1]{\textcolor[rgb]{0.25,0.63,0.44}{#1}}
\newcommand{\DocumentationTok}[1]{\textcolor[rgb]{0.73,0.13,0.13}{\textit{#1}}}
\newcommand{\ErrorTok}[1]{\textcolor[rgb]{1.00,0.00,0.00}{\textbf{#1}}}
\newcommand{\ExtensionTok}[1]{#1}
\newcommand{\FloatTok}[1]{\textcolor[rgb]{0.25,0.63,0.44}{#1}}
\newcommand{\FunctionTok}[1]{\textcolor[rgb]{0.02,0.16,0.49}{#1}}
\newcommand{\ImportTok}[1]{\textcolor[rgb]{0.00,0.50,0.00}{\textbf{#1}}}
\newcommand{\InformationTok}[1]{\textcolor[rgb]{0.38,0.63,0.69}{\textbf{\textit{#1}}}}
\newcommand{\KeywordTok}[1]{\textcolor[rgb]{0.00,0.44,0.13}{\textbf{#1}}}
\newcommand{\NormalTok}[1]{#1}
\newcommand{\OperatorTok}[1]{\textcolor[rgb]{0.40,0.40,0.40}{#1}}
\newcommand{\OtherTok}[1]{\textcolor[rgb]{0.00,0.44,0.13}{#1}}
\newcommand{\PreprocessorTok}[1]{\textcolor[rgb]{0.74,0.48,0.00}{#1}}
\newcommand{\RegionMarkerTok}[1]{#1}
\newcommand{\SpecialCharTok}[1]{\textcolor[rgb]{0.25,0.44,0.63}{#1}}
\newcommand{\SpecialStringTok}[1]{\textcolor[rgb]{0.73,0.40,0.53}{#1}}
\newcommand{\StringTok}[1]{\textcolor[rgb]{0.25,0.44,0.63}{#1}}
\newcommand{\VariableTok}[1]{\textcolor[rgb]{0.10,0.09,0.49}{#1}}
\newcommand{\VerbatimStringTok}[1]{\textcolor[rgb]{0.25,0.44,0.63}{#1}}
\newcommand{\WarningTok}[1]{\textcolor[rgb]{0.38,0.63,0.69}{\textbf{\textit{#1}}}}
\usepackage{longtable,booktabs,array}
\usepackage{calc} % for calculating minipage widths
% Correct order of tables after \paragraph or \subparagraph
\usepackage{etoolbox}
\makeatletter
\patchcmd\longtable{\par}{\if@noskipsec\mbox{}\fi\par}{}{}
\makeatother
% Allow footnotes in longtable head/foot
\IfFileExists{footnotehyper.sty}{\usepackage{footnotehyper}}{\usepackage{footnote}}
\makesavenoteenv{longtable}
\usepackage{graphicx}
\makeatletter
\def\maxwidth{\ifdim\Gin@nat@width>\linewidth\linewidth\else\Gin@nat@width\fi}
\def\maxheight{\ifdim\Gin@nat@height>\textheight\textheight\else\Gin@nat@height\fi}
\makeatother
% Scale images if necessary, so that they will not overflow the page
% margins by default, and it is still possible to overwrite the defaults
% using explicit options in \includegraphics[width, height, ...]{}
\setkeys{Gin}{width=\maxwidth,height=\maxheight,keepaspectratio}
% Set default figure placement to htbp
\makeatletter
\def\fps@figure{htbp}
\makeatother
\ifLuaTeX
  \usepackage{luacolor}
  \usepackage[soul]{lua-ul}
\else
  \usepackage{soul}
\fi
\setlength{\emergencystretch}{3em} % prevent overfull lines
\providecommand{\tightlist}{%
  \setlength{\itemsep}{0pt}\setlength{\parskip}{0pt}}
\setcounter{secnumdepth}{-\maxdimen} % remove section numbering
\ifLuaTeX
  \usepackage{selnolig}  % disable illegal ligatures
\fi
\usepackage{bookmark}
\usepackage[normalem]{ulem}
\IfFileExists{xurl.sty}{\usepackage{xurl}}{} % add URL line breaks if available
\urlstyle{same}
\hypersetup{
  hidelinks,
  pdfcreator={LaTeX via pandoc}}

\author{}
\date{}

\begin{document}

\section{H1}\label{h1}

\subsection{H2}\label{h2}

\subsubsection{H3}\label{h3}

\paragraph{H4}\label{h4}

\subparagraph{H5}\label{h5}

H6

Alternatively, for H1 and H2, an underline-ish style:

\section{Alt-H1}\label{alt-h1}

\subsection{Alt-H2}\label{alt-h2}

Emphasis, aka italics, with \emph{asterisks} or \emph{underscores,
right}.

Strong emphasis, aka bold, with \textbf{asterisks} or
\textbf{underscores}.

Combined emphasis with \textbf{asterisks and \emph{underscores}}.

Strikethrough uses two tildes. \st{Scratch this.}

\begin{enumerate}
\def\labelenumi{\arabic{enumi}.}
\item
  First ordered list item
\item
  Another item

  \begin{itemize}
  \tightlist
  \item
    Unordered sub-list.
  \end{itemize}
\item
  Actual numbers don't matter, just that it's a number

  \begin{enumerate}
  \def\labelenumii{\arabic{enumii}.}
  \tightlist
  \item
    Ordered sub-list
  \end{enumerate}
\item
  And another item.

  You can have properly indented paragraphs within list items. Notice
  the blank line above, and the leading spaces (at least one, but we'll
  use three here to also align the raw Markdown).

  To have a line break without a paragraph, you will need to use two
  trailing spaces. Note that this line is separate, but within the same
  paragraph. (This is contrary to the typical GFM line break behaviour,
  where trailing spaces are not required.)
\end{enumerate}

\begin{itemize}
\tightlist
\item
  Unordered list can use asterisks
\item
  Or minuses
\item
  Or pluses
\end{itemize}

\href{https://www.google.com}{I'm an inline-style link}

\href{https://www.google.com}{I'm an inline-style link with title}

\href{https://www.mozilla.org}{I'm a reference-style link}

\href{../blob/master/LICENSE}{I'm a relative reference to a repository
file}

\href{http://slashdot.org}{You can use numbers for reference-style link
definitions}

Or leave it empty and use the \href{http://www.reddit.com}{link text
itself}.

URLs and URLs in angle brackets will automatically get turned into
links. http://www.example.com or \url{http://www.example.com} and
sometimes example.com (but not on Github, for example).

Some text to show that the reference links can follow later.

\begin{figure}
\centering
\includegraphics{Untitled.jpg}
\caption{Cat}
\end{figure}

Here's our logo (hover to see the title text):

Inline-style:
\includegraphics{https://github.com/adam-p/markdown-here/raw/master/src/common/images/icon48.png}

Reference-style:
\includegraphics{https://github.com/adam-p/markdown-here/raw/master/src/common/images/icon48.png}

Inline \texttt{code} has \texttt{back-ticks\ around} it.

\begin{Shaded}
\begin{Highlighting}[]
\KeywordTok{var}\NormalTok{ s }\OperatorTok{=} \StringTok{"JavaScript syntax highlighting"}\OperatorTok{;}
\FunctionTok{alert}\NormalTok{(s)}\OperatorTok{;}
\end{Highlighting}
\end{Shaded}

\begin{Shaded}
\begin{Highlighting}[]
\NormalTok{s }\OperatorTok{=} \StringTok{"Python syntax highlighting"}
\BuiltInTok{print}\NormalTok{ s}
\end{Highlighting}
\end{Shaded}

\begin{verbatim}
No language indicated, so no syntax highlighting.
But let's throw in a <b>tag</b>.
\end{verbatim}

Here is a simple footnote\footnote{My reference.}.

A footnote can also have multiple lines\footnote{Every new line should
  be prefixed with 2 spaces. This allows you to have a footnote with
  multiple lines.}.

You can also use words, to fit your writing style more
closely\footnote{Named footnotes will still render with numbers instead
  of the text but allow easier identification and linking. This footnote
  also has been made with a different syntax using 4 spaces for new
  lines.}.

Colons can be used to align columns.

\begin{longtable}[]{@{}lcr@{}}
\toprule\noalign{}
Tables & Are & Cool \\
\midrule\noalign{}
\endhead
\bottomrule\noalign{}
\endlastfoot
col 3 is & right-aligned & \$1600 \\
col 2 is & centered & \$12 \\
zebra stripes & are neat & \$1 \\
\end{longtable}

There must be at least 3 dashes separating each header cell. The outer
pipes (\textbar) are optional, and you don't need to make the raw
Markdown line up prettily. You can also use inline Markdown.

\begin{longtable}[]{@{}lll@{}}
\toprule\noalign{}
Markdown & Less & Pretty \\
\midrule\noalign{}
\endhead
\bottomrule\noalign{}
\endlastfoot
\emph{Still} & \texttt{renders} & \textbf{nicely} \\
1 & 2 & 3 \\
\end{longtable}

\begin{quote}
Blockquotes are very handy in email to emulate reply text. This line is
part of the same quote.
\end{quote}

Quote break.

\begin{quote}
This is a very long line that will still be quoted properly when it
wraps. Oh boy let's keep writing to make sure this is long enough to
actually wrap for everyone. Oh, you can \emph{put} \textbf{Markdown}
into a blockquote.
\end{quote}

Definition list

Is something people use sometimes.

Markdown in HTML

Does \emph{not} work \textbf{very} well. Use HTML tags.

Three or more\ldots{}

\begin{center}\rule{0.5\linewidth}{0.5pt}\end{center}

Hyphens

\begin{center}\rule{0.5\linewidth}{0.5pt}\end{center}

Asterisks

\begin{center}\rule{0.5\linewidth}{0.5pt}\end{center}

Underscores

Here's a line for us to start with.

This line is separated from the one above by two newlines, so it will be
a \emph{separate paragraph}.

This line is also a separate paragraph, but\ldots{} This line is only
separated by a single newline, so it's a separate line in the \emph{same
paragraph}.

\end{document}
